\documentclass{article}
\usepackage[utf8]{inputenc}
\usepackage[T1]{fontenc}
\usepackage{amsmath}
\usepackage{amsfonts}
\usepackage{amssymb}
\usepackage[dvipsnames]{xcolor}
\usepackage{enumitem}
\usepackage{titlesec}
\usepackage{graphicx}
\usepackage[total={6.5in, 9in}, heightrounded]{geometry}
\usepackage{hyperref}

\graphicspath{{graphics/}}
\setenumerate[0]{label=\alph*)}
\setlength{\parindent}{0pt}
\setlength{\parskip}{8pt}
\setlength\fboxsep{0pt}
\renewcommand{\baselinestretch}{1.6}
\titleformat{\section}
{\normalfont \Large \bfseries \centering}{}{0pt}{}

\newcommand{\s}[1]{{\color{violet} #1}}

\begin{document}

\Large Name: \rule{2in}{0.15mm} \hfill Homework 5 | Math 342 | Cruz Godar \vspace{4pt} \normalsize

\textit{Due Wednesday of Week 6 at the start of class}

Complete the following problems and submit them as a pdf to Canvas. 8 points are awarded for thoroughly attempting every problem, and I'll select three problems to grade on correctness for 4 points each. Enough work should be shown that there is no question about the mathematical process used to obtain your answers.

\section{Section 6}

In problems 1--5, compute the determinant.

1. $\displaystyle A = \left[\begin{array}{cc} 2& 3 \\ -3& 1 \end{array}\right]$.

2. $\displaystyle B = \left[\begin{array}{ccc} 2& 3& 0 \\ -3& 1& -1 \\ 1& 1& 1 \end{array}\right]$.

3. $\displaystyle C = \left[\begin{array}{ccc} 1& 1& -3 \\ 0& 1& 3 \\ 2& -1& -15 \end{array}\right]$.

4. $\displaystyle D = \left[\begin{array}{ccc} 1& 2& 3 \\ 4& 5& 6 \end{array}\right]$.

5. $\displaystyle E = \left[\begin{array}{cccc} 1& 2& 3& 4 \\ 3& -1& 0& 3 \\ 2& 0& 1& 2\\ -1& -3& -7& 2 \end{array}\right]$.

~\\

6. For each of the matrices $A$--<span class="tex-holder">$E$</span> in problems 1--5, classify it as invertible or noninvertible based on its determinant.

7. Let $A$ be the matrix from problem 1. Sketch a picture of the unit square in $\mathbb{R}^2$ and its image under the linear operator corresponding to $A$. Verify that the area of that image is $|\det A|$ times the area of the unit square (i.e. $1$).

8. We can use the multiplicativity of the determinant to show some nice facts about the determinants of inverse matrices.

\begin{enumerate}

	\item What is $\det I$?

	\item Let $A$ be an invertible matrix. Using part a), find $\det A^{-1}$ in terms of $\det A$.

\end{enumerate}

~\\

9. \textbf{Cramer's Rule} is a method for computing the inverse of a matrix without row reduction. In this problem, we'll work through an example application of it.

\begin{enumerate}

	\item Let $\displaystyle A = \left[\begin{array}{ccc}1& 0& -1 \\ 4& 5& 6 \\ 0& -1& 2\end{array}\right]$ and let $c_{ij}$ be the determinant of the minor given by removing row $i$ and column $j$ from $A$. Find all nine $c_{ij}$ and form a matrix $C$ whose entry in row $i$ and column $j$ is $c_{ij}$.

	\item Form a new matrix $D$ by applying the checkerboard signs to $C$:

\end{enumerate}

\begin{align*}
	\left[\begin{array}{ccc} +& -& + \\ -& +& - \\ +& -& + \end{array}\right].
\end{align*}

\begin{enumerate}

	\item If all went well, the matrix $E = \frac{1}{\det A} D$ should be equal to $(A^{-1})^T$. Compute $E$ and check that it is in fact the transpose of $A^{-1}$ (Note: the matrix $A$ appeared in homework 2).

\end{enumerate}

\section{Eigenvectors and Eigenvalues}

Let $A$ be an $n \times n$ matrix. When $A$ only scales a nonzero vector and doesn't multiply it --- i.e $A\vec{v} = \lambda\vec{v}$ for a vector $\vec{v}$ and a constant $\lambda$ --- we say that $\vec{v}$ is an \textbf{eigenvector} of $A$ with \textbf{eigenvalue} $\lambda$. You'll see more on these in the next course if you take it, but for now, we'll work through a few basic examples.

10. Let $\displaystyle A = \left[\begin{array}{cc} 2& -1 \\ 3& -2 \end{array}\right]$. Show that $\vec{v_1} = \left[\begin{array}{c} 1 \\ 3 \end{array}\right]$ and $\vec{v_2} = \left[\begin{array}{c} 1 \\ 1 \end{array}\right]$ are eigenvectors of $A$ and find their eigenvalues.

11. Let $B = \left[\begin{array}{cc} -1& 2 \\ 0& -3 \end{array}\right]$.

\begin{enumerate}

	\item If $B\vec{v} = \lambda\vec{v}$ for a nonzero vector $\vec{v}$, then $B\vec{v} = \lambda I \vec{v}$, so $(B - \lambda I)\vec{v} = \vec{0}$. That means $B - \lambda I$ is not one-to-one, so $\det (B - \lambda I) = 0$ (the left side is called the \textbf{characteristic polynomial} of $B$). Find that determinant and solve it for $\lambda$.

	\item The values of $\lambda$ in part a) are the eigenvalues of $B$. For each value of $\lambda$, we want to solve $(B - \lambda I)\vec{v} = 0$, so augment $B - \lambda I$ with $\vec{0}$ and row reduce. In total, what are the eigenvectors and eigenvalues of $B$?

\end{enumerate}

12. Let $C = \left[\begin{array}{ccc} 2& 2& -2 \\ -3& 7& 3 \\ -5& 5& 5 \end{array}\right]$. Find the eigenvectors and eigenvalues of $C$ in the same manner as the previous problem.

~\\

13. Let $A$ be an $n \times n$ matrix with eigenvalues $\lambda_1, \lambda_2, ..., \lambda_n$.

\begin{enumerate}

	\item What is the characteristic polynomial $\det(A - \lambda I)$ of $A$?

	\item By setting $\lambda = 0$ in part a), express $\det A$ in terms of the $\lambda_i$.

\end{enumerate}

~\\

14. One application of eigenvalues is to systems of differential equations. If $\vec{v_1}$ and $\vec{v_2}$ are eigenvectors of a $2 \times 2$ matrix $A$ with eigenvalues $\lambda_1$ and $\lambda_2$, then the solution to the system

$$
	\left[\begin{array}{c} x'(t) \\ y'(t) \end{array}\right] = A\left[\begin{array}{c} x(t) \\ y(t) \end{array}\right]
$$

is

$$
	\left[\begin{array}{c} x \\ y \end{array}\right] = c_1 e^{\lambda_1 t} \vec{v_1} + c_2 e^{\lambda_2 t} \vec{v_2}
$$

for any value of $c_1$ and $c_2$.

\begin{enumerate}

	\item Write the general solutions to the systems

\end{enumerate}

\begin{align*}
	x'(t) &= 2x(t) - y(t)\\
	y'(t) &= 3x(t) - 2y(t)
\end{align*}

<span style="width: 32px"></span>and

\begin{align*}
	x'(t) &= -x(t) + 2y(t)\\
	y'(t) &= -3y(t).
\end{align*}

\begin{enumerate}

	\item We can plot solutions to systems of differential equations as \textbf{vector fields}: every point $(x, y)$ has a velocity $(x', y')$, so if we fill an area with particles and move them according to that velocity, we can see the entire effect of the system. Using \href{https://cruzgodar.com/applets/vector-fields}{a vector field applet} with the generating functions <code>(2x - y, 2x - 2y)</code> and <code>(-x + 2y, -3y)</code>, plot the systems from part a).

\end{enumerate}


\end{document}