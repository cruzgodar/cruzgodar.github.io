\documentclass{article}
\usepackage[utf8]{inputenc}
\usepackage[T1]{fontenc}
\usepackage{amsmath}
\usepackage{amsfonts}
\usepackage{amssymb}
\usepackage[dvipsnames]{xcolor}
\usepackage{enumitem}
\usepackage{titlesec}
\usepackage{graphicx}
\usepackage[total={6.5in, 9in}, heightrounded]{geometry}
\usepackage{hyperref}

\graphicspath{{graphics/}}
\setenumerate[0]{label=\alph*)}
\setlength{\parindent}{0pt}
\setlength{\parskip}{8pt}
\setlength\fboxsep{0pt}
\renewcommand{\baselinestretch}{1.6}
\titleformat{\section}
{\normalfont \Large \bfseries \centering}{}{0pt}{}

\newcommand{\s}[1]{{\color{violet} #1}}

\begin{document}

\Large Name: \rule{2in}{0.15mm} \hfill Homework 1 | Math 341 | Cruz Godar \vspace{4pt} \normalsize

\textit{Due Wednesday of Week 2 at the start of class}

Complete the following problems and submit them as a pdf to Canvas. 8 points are awarded for thoroughly attempting every problem, and I'll select three problems to grade on correctness for 4 points each. Enough work should be shown that there is no question about the mathematical process used to obtain your answers.

\section{Section 1}

In problems 1--3, write the system in the form $A\vec{x} = \vec{b}$ for a matrix $A$ and vector $\vec{b}$ of constants and a vector $\vec{x}$ of variables.

1.

\begin{align*}
	x_1 &= 2\\
	2x_1 - x_2 &= 3.
\end{align*}

2.

\begin{align*}
	2x_3 - x_2 &= 0\\
	x_1 &= x_3 - x_2 + 1
\end{align*}

3.

\begin{align*}
	x + y - z &= x\\
	x + 2y - 1 &= 2z\\
	x - z + 1 &= y
\end{align*}

~\\

In problems 4--8, evaluate the product.

4. $\displaystyle \left[\begin{array}{cc}3& 0 \\ 6& -2\end{array}\right]\left[\begin{array}{c}1 \\ -1\end{array}\right].$

5. $\displaystyle \left[\begin{array}{ccc}1& 2& 3\end{array}\right]\left[\begin{array}{c}4 \\ 5 \\ 6\end{array}\right].$

6. $\displaystyle \left[\begin{array}{c}4 \\ 5 \\ 6\end{array}\right]\left[\begin{array}{ccc}1& 2& 3\end{array}\right].$

7. $\displaystyle \left[\begin{array}{c}1 \\ 2 \\ 3\end{array}\right]\left[\begin{array}{c}4 \\ 5 \\ 6\end{array}\right].$

8. $\displaystyle \left[\begin{array}{cc}1& 0 \\ 0& 1 \\ 1& 1\end{array}\right]\left[\begin{array}{cccc}1& -1& 1& -1 \\ -1& 1& -1& 1\end{array}\right].$

~\\

9. Let $A$ be an $n \times n$ matrix with entries $a_{ij}$.

\begin{enumerate}

	\item For the products $AI$ and $IA$ to make sense, what dimension must $I$ have?

	\item The $i$th row of $A$ is $\left[\begin{array}{cccc}a_{i1}& a_{i2}& \cdots& a_{in}\end{array}\right]$. If the $j$th column of $I$ is denoted $\vec{e_j}$, what is the entry in row $i$ and column $j$ of $AI$? Your answer should be in terms of $i$ and $j$.

	\item What does part b) imply $AI$ is equal to? Why does this make sense in the context of function composition?

\end{enumerate}

~\\

10. Let $A$ be an $m \times n$ matrix with entries $a_{ij}$.

\begin{enumerate}

	\item There is a row vector $\vec{x}$ and a column vector $\vec{y}$ such that $\vec{x}A\vec{y} = a_{11}$. What are they?

	\item What about vectors $\vec{x}$ and $\vec{y}$ such that $\vec{x}A\vec{y} = a_{ij}$? Your answer should be in terms of $i$ and $j$.

\end{enumerate}

~\\

11. State whether each part is true or false. If true, briefly justify why, and if false, provide a small counterexample.

\begin{enumerate}

	\item A system with $3$ equations and $2$ unknowns always has at least one solution.

	\item If the product $AB$ is defined, then $A$ and $B$ have the same number of rows.

	\item If $A$ is a $2 \times 2$ matrix so that $A\vec{x} = \vec{x}$ for \textit{every} vector $\vec{x}$, then $A = I_2$. Hint: try plugging in specific values of $x_1$ and $x_2$, like $0$ and $1$.

\end{enumerate}

~\\

In problems 12--14, we'll show that many of the nice properties of multiplication of numbers don't hold for matrices.

12. With real numbers $x$ and $y$, it must be the case that $xy = yx$, but this isn't true with matrices. Define matrices $A$ and $B$ by

\begin{align*}
	A = \left[\begin{array}{ccc}1& 2& 3 \\ 4& 5& 6 \\ 7& 8& 9\end{array}\right], \quad B = \left[\begin{array}{ccc}1& 0& 2 \\ -2& 3& 0 \\ 0& 4& 1\end{array}\right]
\end{align*}

Compute $AB$ and $BA$ and show that they're different.

13. With real numbers $x$, $y$, and $z$ where $x \neq 0$ and $xy = xz$, it's always the case that $y = z$, but this also isn't true for matrices. Define matrices $A$, $B$, and $C$ by

\begin{align*}
	A = \left[\begin{array}{cc}1& 2 \\ 0& 0\end{array}\right], \quad B = \left[\begin{array}{cc}1& 2 \\ 3& 4\end{array}\right], \quad C = \left[\begin{array}{cc}5& 4 \\ 1& 3\end{array}\right]
\end{align*}

Show that $AB = AC$, despite $B \neq C$.

14. With real numbers $x$ and $y$ where $xy = 0$, either $x = 0$ or $y = 0$. Unfortunately, this also isn't the case for matrices. Define $A$ and $B$ by

\begin{align*}
	A = \left[\begin{array}{ccc}1& 0& 0 \\ 0& 0& 0 \\ 0& 0& 0\end{array}\right], \quad B = \left[\begin{array}{ccc}0& 0& 0 \\ 0& 2& 3 \\ 0& 4& 5\end{array}\right].
\end{align*}

Show that $AB = 0$ (the matrix of all zeros), even though both $A$ and $B$ are nonzero.


\end{document}