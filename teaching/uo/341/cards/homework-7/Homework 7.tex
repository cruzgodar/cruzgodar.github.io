\documentclass{article}
\usepackage[utf8]{inputenc}
\usepackage[T1]{fontenc}
\usepackage{amsmath}
\usepackage{amsfonts}
\usepackage{amssymb}
\usepackage[dvipsnames]{xcolor}
\usepackage{enumitem}
\usepackage{titlesec}
\usepackage{graphicx}
\usepackage[total={6.5in, 9in}, heightrounded]{geometry}
\usepackage{hyperref}

\graphicspath{{graphics/}}
\setenumerate[0]{label=\alph*)}
\setlength{\parindent}{0pt}
\setlength{\parskip}{8pt}
\setlength\fboxsep{0pt}
\renewcommand{\baselinestretch}{1.6}
\titleformat{\section}
{\normalfont \Large \bfseries \centering}{}{0pt}{}

\newcommand{\s}[1]{{\color{violet} #1}}

\begin{document}

\Large Name: \rule{2in}{0.15mm} \hfill Homework 7 | Math 341 | Cruz Godar \vspace{4pt} \normalsize

\textit{Due Wednesday of Week 8 at the start of class}

Complete the following problems and submit them as a pdf to Canvas. 8 points are awarded for thoroughly attempting every problem, and I'll select three problems to grade on correctness for 4 points each. Enough work should be shown that there is no question about the mathematical process used to obtain your answers.

\section{Section 8}

In problems 1--6, find \textbf{two different} bases $\mathcal{B}$ and $\mathcal{C}$ for the vector space $V$ and use them to find $\dim V$. Then with the given vector $\vec{v}$, find $[\vec{v}]_\mathcal{C}$.

1. $V = \mathbb{R}^3$, and $\vec{v} = \left[\begin{array}{c} 1 \\ 2 \\ 3 \end{array}\right]$.

2. $V = M_{2 \times 2}(\mathbb{R})$, and $\vec{v} = \left[\begin{array}{cc} 2& -1 \\ 0& 1 \end{array}\right]$.

3. $V = \mathcal{L}(\mathbb{R}^2, \mathbb{R}^2)$, and $\vec{v}:\mathbb{R}^2 \to \mathbb{R}^2$ is defined by $\vec{v}\left( \left[\begin{array}{c} x \\ y \end{array}\right] \right) = \left[\begin{array}{c} 2x \\ x + y \end{array}\right]$. (Hint: your answers to the previous problem may help.)

4. $V$ is the subspace of $\mathbb{R}^4$ of vectors $\left[\begin{array}{c} x \\ y \\ z \\ w \end{array}\right]$ satisfying $x + y - w = 0$, and $\vec{v} = \left[\begin{array}{c} 1 \\ 1 \\ 3 \\ 2 \end{array}\right]$.

5. $V = \mathbb{R}[x]$, and $\vec{v} = (x^2 - 2)^2$.

6. $V = \operatorname{span}\{\cos(x), \sin(x)\}$, and $\vec{v} = \sin\left(x + \frac{\pi}{4} \right)$. (Hint: the sum and difference formulas for $\sin$ and $\cos$ may be helpful.)

~\\

In problems 7--9, find a matrix for the linear transformation $T : V \to W$ by choosing bases $\mathcal{B}$ for $V$ and $\mathcal{C}$ for $W$. Then use the matrix to evaluate $T(\vec{v})$ for the given vector $v$.

7. $V = \mathbb{R}^3$ and $W = \mathbb{R}$, $T: V \to W$ is a transformation for which

$$
	T\left( \left[\begin{array}{c} 1 \\ 0 \\ 1 \end{array}\right] \right) = 1 \qquad T\left( \left[\begin{array}{c} 2 \\ 0 \\ 1 \end{array}\right] \right) = 2 \qquad T\left( \left[\begin{array}{c} 0 \\ -1 \\ 0 \end{array}\right] \right) = -1,
$$

and $\vec{v} = \left[\begin{array}{c} 1 \\ 1 \\ 1 \end{array}\right]$.

8. $V$ and $W$ are both the subspace of $\mathbb{R}[x]$ of polynomials with degree at most $2$, $T: V \to W$ is a transformation for which

$$
	T(1) = x \qquad T(x^2 + x) = 2x \qquad T(x^2) = x^2,
$$

and $\vec{v} = x^2 - x - 1$.

9. $V = M_{2 \times 2}(\mathbb{R})$, $W = \mathbb{R}^2$, $T: V \to W$ is a transformation for which

\begin{align*}
	T\left(\left[\begin{array}{cc} 1& 0 \\ 0& 1 \end{array}\right]\right) = \left[\begin{array}{c} 2 \\ 0 \end{array}\right] \qquad T\left(\left[\begin{array}{cc} 0& 1 \\ 1& 0 \end{array}\right]\right) = \left[\begin{array}{c} 0 \\ 2 \end{array}\right] \qquad T\left(\left[\begin{array}{cc} 1& 1 \\ 0& 0 \end{array}\right]\right) = \left[\begin{array}{c} 1 \\ 1 \end{array}\right] \qquad T\left(\left[\begin{array}{cc} 0& 0 \\ 0& 1 \end{array}\right]\right) = \left[\begin{array}{c} 1 \\ 0 \end{array}\right],
\end{align*}

and $\vec{v} = \left[\begin{array}{cc} 1& 2 \\ 3& 4 \end{array}\right]$.


\end{document}