\documentclass{article}
\usepackage[utf8]{inputenc}
\usepackage[T1]{fontenc}
\usepackage{amsmath}
\usepackage{amsfonts}
\usepackage{amssymb}
\usepackage[dvipsnames]{xcolor}
\usepackage{enumitem}
\usepackage{titlesec}
\usepackage{graphicx}
\usepackage[total={6.5in, 9in}, heightrounded]{geometry}
\usepackage{hyperref}

\graphicspath{{graphics/}}
\setenumerate[0]{label=\alph*)}
\setlength{\parindent}{0pt}
\setlength{\parskip}{8pt}
\setlength\fboxsep{0pt}
\renewcommand{\baselinestretch}{1.6}
\titleformat{\section}
{\normalfont \Large \bfseries \centering}{}{0pt}{}

\newcommand{\s}[1]{{\color{violet} #1}}

\begin{document}

\Large Name: \rule{2in}{0.15mm} \hfill Homework 3 | Math 341 | Cruz Godar \vspace{4pt} \normalsize

\textit{Due Wednesday of Week 4 at the start of class}

Complete the following problems and submit them as a pdf to Canvas. 8 points are awarded for thoroughly attempting every problem, and I'll select three problems to grade on correctness for 4 points each. Enough work should be shown that there is no question about the mathematical process used to obtain your answers.

\section{Section 3}

1. Let $\displaystyle \vec{v} = \left[\begin{array}{c} 1 \\ 3 \end{array}\right]$ and $\displaystyle \vec{w} = \left[\begin{array}{c} -1 \\ 1 \end{array}\right]$.

\begin{enumerate}

	\item Draw $\vec{v}$ and $\vec{w}$ in the plane.

	\item Draw $2\vec{v} + \vec{w}$ geometrically and verify that it matches the algebraic definition (i.e. adding the entries of $\vec{v}$ and $\vec{w}$).

	\item Draw a vector linearly dependent with $\vec{v}$, but linearly independent with $\vec{w}$.

	\item Draw a vector linearly dependent with both $\vec{v}$ and $\vec{w}$, but not with either of them alone.

\end{enumerate}

~\\

In problems 2--5, determine if the vectors are linearly dependent or independent. If they are dependent, find a linear combination equal to $\vec{0}$.

2. $\vec{v_1} = \left[\begin{array}{c} 2 \\ 3 \end{array}\right]$, $\vec{v_2} = \left[\begin{array}{c} 4 \\ 5 \end{array}\right]$.

3. $\vec{v_1} = \left[\begin{array}{c} 1 \\ 2 \\ 3 \end{array}\right]$, $\vec{v_2} = \left[\begin{array}{c} 1 \\ 0 \\ 2 \end{array}\right]$, $\vec{v_3} = \left[\begin{array}{c} 1 \\ 4 \\ 4 \end{array}\right]$.

4. $\vec{v_1} = \left[\begin{array}{c} 1 \\ 1 \\ 2 \end{array}\right]$, $\vec{v_2} = \left[\begin{array}{c} 2 \\ 0 \\ 3 \end{array}\right]$, $\vec{v_3} = \left[\begin{array}{c} 4 \\ - 1 \\ 2 \end{array}\right]$

5. $\vec{v_1} = \left[\begin{array}{c} 1 \\ 5 \\ 6 \\ -10 \\ \frac{17}{2} \end{array}\right]$, $\vec{v_2} = \left[\begin{array}{c} 3 \\ 1 \\ 7 \\ -100 \\ 0\end{array}\right]$, $\vec{v_3} = \left[\begin{array}{c} 0 \\ 0 \\ 0 \\ 0 \\ 0 \end{array}\right]$, $\vec{v_4} = \left[\begin{array}{c} 4 \\ 1 \\ 2 \\ 0 \\ 89 \end{array}\right]$.

~\\

In problems 6--7, express $\vec{v}$ as a linear combination of the $\vec{u}_i$ or show it's impossible.

6. $\vec{v} = \left[\begin{array}{c} 1 \\ 2 \\ 3 \end{array}\right]$, $\vec{u_1} = \left[\begin{array}{c} 1 \\ 2 \\ 2 \end{array}\right]$, $\vec{u_2} = \left[\begin{array}{c} 2 \\ -1 \\ 0 \end{array}\right]$, $\vec{u_3} = \left[\begin{array}{c} 3 \\ 1 \\ 1 \end{array}\right]$.

7. $\vec{v} = \left[\begin{array}{c} 1 \\ 2 \\ 3 \end{array}\right]$, $\vec{u_1} = \left[\begin{array}{c} -1 \\ 3 \\ 0 \end{array}\right]$, $\vec{u_2} = \left[\begin{array}{c} 1 \\ 1 \\ 1 \end{array}\right]$, $\vec{u_3} = \left[\begin{array}{c} 2 \\ 6 \\ 3 \end{array}\right]$.

~\\

8. Let $\vec{w_1} = \left[\begin{array}{c} 1 \\ 1 \\ 0 \end{array}\right]$, $\vec{w_2} = \left[\begin{array}{c} 2 \\ 0 \\ -1 \end{array}\right]$, $\vec{w_3} = \left[\begin{array}{c} 0 \\ 2 \\ 1 \end{array}\right]$.

\begin{enumerate}

	\item Show that $\vec{w_1}$, $\vec{w_2}$, and $\vec{w_3}$ are linearly dependent.

	\item Show that just $\vec{w_1}$ and $\vec{w_2}$ on their own are linearly independent. (Hint: You should be able to modify the last step in the previous part to get this result without starting over).

	\item Using the previous two parts, write a sentence explaining why

\end{enumerate}

$$
	\operatorname{span}\{\vec{w_1}, \vec{w_2}\} = \operatorname{span}\{\vec{w_1}, \vec{w_2}, \vec{w_3}\}.
$$

\begin{enumerate}

	\item The span of $\vec{w_1}$ and $\vec{w_2}$ is the set consisting of all $\vec{w} = c_1\vec{w_1} + c_2\vec{w_2}$. Renaming $c_1 = u$ and $c_2 = v$, write the generic vector $\vec{w}$ in the span. Your answer should depend on $u$ and $v$.

	\item \href{https://www.math3d.org/8Adpukc7b}{Math3D} is a capable 3D grapher that can handle parametric surfaces. Open the linked example and replace the span expression with the one you found in the previous part --- if all went well, you should see the three vectors lying \textit{in} that plane. This plane is the span, and the fact that the three vectors are contained in a two-dimensional surface is their linear dependence.

\end{enumerate}

\section{Section 4}

9. Linear transformations are related to typical linear functions like $y = mx + b$, but they're not quite the same. For example, the function $f : \mathbb{R} \to \mathbb{R}$ defined by $f(x) = 2x + 3$ is a linear function, but not a linear transformation. Pick two inputs $a$ and $b$ and show that $f(a + b) \neq f(a) + f(b)$.

~\\

In problems 10--12, find the matrix for the linear transformation $T$ and use it to evaluate $T(\vec{v})$.

10. $T\left( \left[\begin{array}{c} 1 \\ 1 \\ 0 \end{array}\right] \right) = \left[\begin{array}{c} 1 \\ 1 \end{array}\right]$, $T\left( \left[\begin{array}{c} 0 \\ 1 \\ 0 \end{array}\right] \right) = \left[\begin{array}{c} 3 \\ 4 \end{array}\right]$, $T\left( \left[\begin{array}{c} 0 \\ 0 \\ 3 \end{array}\right] \right) = \left[\begin{array}{c} 3 \\ 1 \end{array}\right]$, and $\vec{v} = \left[\begin{array}{c} 2 \\ 3 \\ 2 \end{array}\right]$.

11. $T\left( \left[\begin{array}{c} 1 \\ 1 \end{array}\right] \right) = \left[\begin{array}{c} 1 \\ 1 \end{array}\right]$, $T\left( \left[\begin{array}{c} 1 \\ -1 \end{array}\right] \right) = \left[\begin{array}{c} -1 \\ 1 \end{array}\right]$, and $\vec{v} = \left[\begin{array}{c} 2 \\ 3 \end{array}\right]$.

12. $T\left( \left[\begin{array}{c} 0 \\ 1 \end{array}\right] \right) = 2$, $T\left( \left[\begin{array}{c} 1 \\ 3 \end{array}\right] \right) = 7$, and $\vec{v} = \left[\begin{array}{c} 2 \\ 1 \end{array}\right]$.

~\\

13. Let $T : \mathbb{R}^3 \to \mathbb{R}^3$ be the linear transformation given by

$$
	T \left( \left[\begin{array}{c} x \\ y \\ z \end{array}\right] \right) = \left[\begin{array}{c} 2x + 3y \\ x + 2y \\ 3x + z \end{array}\right].
$$

\begin{enumerate}

	\item Express $T$ as a $3 \times 3$ matrix $A$.

	\item Find $A^{-1}$.

	\item Write a linear transformation $S$ whose matrix is $A^{-1}$.

	\item Since $A^{-1}A = I$ and matrix multiplication is equivalent to function composition, we should expect $S \circ T = id$, the identity function. Evaluate $S \circ T$ as functions by using the output of $T$ as the input to $S$, and show this is in fact the case.

\end{enumerate}


\end{document}