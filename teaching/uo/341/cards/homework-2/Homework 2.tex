\documentclass{article}
\usepackage[utf8]{inputenc}
\usepackage[T1]{fontenc}
\usepackage{amsmath}
\usepackage{amsfonts}
\usepackage{amssymb}
\usepackage[dvipsnames]{xcolor}
\usepackage{enumitem}
\usepackage{titlesec}
\usepackage{graphicx}
\usepackage[total={6.5in, 9in}, heightrounded]{geometry}
\usepackage{hyperref}

\graphicspath{{graphics/}}
\setenumerate[0]{label=\alph*)}
\setlength{\parindent}{0pt}
\setlength{\parskip}{8pt}
\setlength\fboxsep{0pt}
\renewcommand{\baselinestretch}{1.6}
\titleformat{\section}
{\normalfont \Large \bfseries \centering}{}{0pt}{}

\newcommand{\s}[1]{{\color{violet} #1}}

\begin{document}

\Large Name: \rule{2in}{0.15mm} \hfill Homework 2 | Math 342 | Cruz Godar \vspace{4pt} \normalsize

\textit{Due Wednesday of Week 3 at the start of class}

Complete the following problems and submit them as a pdf to Canvas. 8 points are awarded for thoroughly attempting every problem, and I'll select three problems to grade on correctness for 4 points each. Enough work should be shown that there is no question about the mathematical process used to obtain your answers.

\section{Section 2}

In problems 1--6, write the system as an augmented matrix and row reduce it, indicating every elementary row operation you perform. Then use the reduced matrix to write the solution to the system.

1.

\begin{align*}
	3x - y &= 14\\
	4x + 2y &= 2.
\end{align*}

2.

\begin{align*}
	x + 2z &= 8\\
	-x + 2y + 6z &= 6\\
	4x + y + 3z &= 21.
\end{align*}

3.

\begin{align*}
	3x_1 + x_2 + x_3 &= 1\\
	x_2 + x_3 &= -4\\
	6x_1 + 5x_2 + 8x_3 &= -10.
\end{align*}

4.

\begin{align*}
	x_1 + x_2 - 4x_3 &= -11\\
	-3x_1 + 2x_3 &= 3\\
	2x_1 + 2x_2 + 2x_3 &= 8\\
	-x_1 + 2x_2 &= 1.
\end{align*}

5.

\begin{align*}
	a + c &= 2\\
	b + d &= 1\\
	a + 2b + 3c + 4d &= 10\\
	4a + 3b + 2c + d &= 5.
\end{align*}

6.

\begin{align*}
	x + y + 4z &= 4\\
	x - y + 2w &= 6\\
	-x + 5y + 8z - 6w &= -10\\
	x + 2z + w &= 5
\end{align*}

~\\

In problems 7--11, invert the matrix.

7. $\displaystyle \left[\begin{array}{cc}4& 1 \\ -1& 2\end{array}\right].$

8. $\displaystyle \left[\begin{array}{ccc}1& 0& -1 \\ 4& 5& 6 \\ 0& -1& 2\end{array}\right].$

9. $\displaystyle \left[\begin{array}{ccc}2& 1& 3 \\ 3& -1& 2 \\ 1& 0& 1\end{array}\right].$

10. $\displaystyle \left[\begin{array}{c}4\end{array}\right].$

11. $\displaystyle \left[\begin{array}{cccc}1& 0& 0& 0 \\ 0& 1& 0& 1 \\ 0& 0& 1& 0 \\ 1& 0& 1& 1\end{array}\right]$

~\\

12. Let's investigate the elementary row operations a bit more. For a $3 \times 3$ matrix $A$, find the following:

\begin{enumerate}

	\item A matrix $S_{1, 2}$ so that $S_{1, 2}A$ is the equal to $A$, but with rows $1$ and $2$ swapped. Similarly, find $S_{1, 3}$ and $S_{2, 3}$.

	\item A matrix $M_{1, c}$ so that $M_{1, c}A$ is same as $A$, but with row $1$ multiplied by $c$. Similarly, find $M_{2, c}$ and $M_{3, c}$.

	\item A matrix $P_{1, 2, c}$ so that $P_{1, 2, c}A$ is same as $A$, but with $c$ times row $2$ added to row $1$. Similarly, find $P_{1, 3, c}$, $P_{2, 3, c}$, as well as $P_{2, 1, c}$, $P_{3, 1, c}$, and $P_{3, 2, c}$.

\end{enumerate}

~\\

13. Let $\displaystyle A = \left[\begin{array}{cc}a& b \\ c& d\end{array}\right]$ be a generic $2 \times 2$ matrix. Find $A^{-1}$ in terms of $a$, $b$, $c$, and $d$.


\end{document}