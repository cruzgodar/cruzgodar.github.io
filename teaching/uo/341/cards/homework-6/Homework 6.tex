\documentclass{article}
\usepackage[utf8]{inputenc}
\usepackage[T1]{fontenc}
\usepackage{amsmath}
\usepackage{amsfonts}
\usepackage{amssymb}
\usepackage[dvipsnames]{xcolor}
\usepackage{enumitem}
\usepackage{titlesec}
\usepackage{graphicx}
\usepackage[total={6.5in, 9in}, heightrounded]{geometry}
\usepackage{hyperref}

\graphicspath{{graphics/}}
\setenumerate[0]{label=\alph*)}
\setlength{\parindent}{0pt}
\setlength{\parskip}{8pt}
\setlength\fboxsep{0pt}
\renewcommand{\baselinestretch}{1.6}
\titleformat{\section}
{\normalfont \Large \bfseries \centering}{}{0pt}{}

\newcommand{\s}[1]{{\color{violet} #1}}

\begin{document}

\Large Name: \rule{2in}{0.15mm} \hfill Homework 6 | Math 341 | Cruz Godar \vspace{4pt} \normalsize

\textit{Due Wednesday of Week 7 at the start of class}

Complete the following problems and submit them as a pdf to Canvas. 8 points are awarded for thoroughly attempting every problem, and I'll select three problems to grade on correctness for 4 points each. Enough work should be shown that there is no question about the mathematical process used to obtain your answers.

\section{Section 7}

In problems 1--5, determine if the set $X$ is a subspace of the vector space $V$. If it is, show that $X$ is closed under addition and scalar multiplication and contains the zero vector, and if not, give an example showing one of those three fails.

1. $X$ is the set of vectors of the form $\left[\begin{array}{c} x \\ y \\ 1 \end{array}\right]$ for real numbers $x$ and $y$, and $V = \mathbb{R}^3$.

2. $X = \operatorname{span} \{ \cos(x), \sin(x) \}$, and $V = C^0(\mathbb{R})$.

3. $X$ is the set of matrices of the form $\left[\begin{array}{cc} a& 0 \\ 0& b \end{array}\right]$, and $V = M_{2 \times 2}(\mathbb{R})$.

4. $X$ is the set of linear transformations $T : \mathbb{R}^2 \to \mathbb{R}^2$ such that $T \left( \left[\begin{array}{c} 1 \\ 1 \end{array}\right] \right) = \left[\begin{array}{c} 1 \\ 1 \end{array}\right]$, and $V = \mathcal{L}(\mathbb{R}^2, \mathbb{R}^2)$.

5. $X$ is the set of linear transformations $T : \mathbb{R}^3 \to \mathbb{R}^2$ with

$$
	\ker T = \operatorname{span}\left\{ \left[\begin{array}{c} 1 \\ 1 \\ 0 \end{array}\right], \left[\begin{array}{c} 0 \\ 0 \\ 1 \end{array}\right] \right\},
$$

and $V = \mathcal{L}(\mathbb{R}^3, \mathbb{R}^2)$.

~\\

In problems 6--10, determine if the given function $T$ is a linear transformation. If it is, show that $T$ splits across addition and scalar multiplication, and if it is not, give an example showing one of those two things fails. If $T$ is a linear transformation, also find $\ker T$ and write it as a span of vectors.

6. $T : \mathbb{R}^3 \to \mathbb{R}^2$, defined by

$$
	T \left( \left[\begin{array}{c} x \\ y \\ z \end{array}\right] \right) = \left[\begin{array}{c} x + y + z \\ 2x - y - z \end{array}\right].
$$

7. $T : \mathbb{R}[x] \to \mathbb{R}$ given by $T(p(x)) = p''(x)$.

8. $T : \mathbb{R}^\mathbb{R} \to \mathbb{R}$ given by $T(f) = f(0)$. In this problem, just describe the kernel in words rather than as a span.

9. $T : M_{2 \times 2}(\mathbb{R}) \to \mathbb{R}$ given by $T(A) = \det A$.

10. $T : \mathcal{L}(\mathbb{R}^2, \mathbb{R}^2) \to \mathbb{R}^2$ given by $T(S) = S\left( \left[\begin{array}{c} 1 \\ 1 \end{array}\right] \right)$.


\end{document}