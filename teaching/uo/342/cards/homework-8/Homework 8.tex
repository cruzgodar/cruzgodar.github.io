\documentclass{article}
\usepackage[utf8]{inputenc}
\usepackage[T1]{fontenc}
\usepackage{amsmath}
\usepackage{amsfonts}
\usepackage{amssymb}
\usepackage[dvipsnames]{xcolor}
\usepackage{enumitem}
\usepackage{titlesec}
\usepackage{graphicx}
\usepackage[total={6.5in, 9in}, heightrounded]{geometry}
\usepackage{hyperref}

\graphicspath{{graphics/}}
\setenumerate[0]{label=\alph*)}
\setlength{\parindent}{0pt}
\setlength{\parskip}{8pt}
\setlength\fboxsep{0pt}
\renewcommand{\baselinestretch}{1.6}
\titleformat{\section}
{\normalfont \Large \bfseries \centering}{}{0pt}{}

\newcommand{\s}[1]{{\color{violet} #1}}

\begin{document}

\Large Name: \rule{2in}{0.15mm} \hfill Homework 8 | Math 342 | Cruz Godar \vspace{4pt} \normalsize

\textit{Due Wednesday of Week 10 at the start of class}

Complete the following problems and submit them as a pdf to Canvas. 8 points are awarded for thoroughly attempting every problem, and I'll select three problems to grade on correctness for 4 points each. Enough work should be shown that there is no question about the mathematical process used to obtain your answers.

~\\

In problems 1--5, do the following:

\begin{enumerate}

	\item Find the eigenvalues of $A$.

	\item Find the corresponding eigenvectors and generalized eigenvectors of $A$.

	\item Determine if $A$ is diagonalizable. If it is, write $A = BDB^{-1}$ for a diagonal matrix $D$, and if not, write $A = BJB^{-1}$ for a matrix $J$ in Jordan normal form (you don't need to invert $B$).

	\item If $A$ is diagonalizable, determine if there is an orthonormal basis of eigenvectors; if so, find it.

	\item If the eigenvalues of $A$ are distinct, find the general solution to the system of differential equations $\vec{x}' = A\vec{x}$.

\end{enumerate}

1. $A = \left[\begin{array}{cc} 3& 1 \\ -1& 1 \end{array}\right]$.

2. $A = \left[\begin{array}{ccc} 2& 1& 0 \\ 0& 1& 0 \\ 1& -1& 3 \end{array}\right]$.

3. $A = \left[\begin{array}{ccc} -3& -1& -3 \\ -8& -3& -8 \\ 4& 1& 3 \end{array}\right]$.

4. $A = \left[\begin{array}{ccc} -9& -10& -10 \\ 4& 4& 3 \\ 1& 2& 3 \end{array}\right]$.

5. $A = \left[\begin{array}{cccc} 1& 3& 0& 0 \\ 3& 1& 0& 0 \\ 0& 0& 1& -3 \\ 0& 0& -3& 1 \end{array}\right]$.

~\\

In problems 6--8, do the following:

\begin{enumerate}

	\item Find a singular value decomposition $A = U \Sigma V^T$.

	\item Find a least-squares solution to $A\vec{x} = \vec{b}$.

	\item Determine if the least-squares solution is unique.

\end{enumerate}

6. $A = \left[\begin{array}{cc} 1& 2 \\ 2& 1 \\ 1& 1 \end{array}\right]$ and $\vec{b} = \left[\begin{array}{c} 1 \\ 2 \\ 2 \end{array}\right]$.

7. $A = \left[\begin{array}{ccc} 3& 1& 4 \\ 2& 0& 1 \end{array}\right]$ and $\vec{b} = \left[\begin{array}{c} 1 \\ -2 \end{array}\right]$.

8. $A = \left[\begin{array}{ccc} 1& -1& 0 \\ 1& 0& 1 \\ 2& -1& 1 \end{array}\right]$ and $\vec{b} = \left[\begin{array}{c} 1 \\ 2 \\ -1 \end{array}\right]$.

~\\

In problems 9--12, do the following:

\begin{enumerate}

	\item Find an orthonormal basis for the given inner product space $X$, and then extend it to an orthonormal basis for $V$.

	\item Find the orthogonal decomposition of the vector $\vec{v}$ as $\vec{v} = \vec{x} + \vec{x}'$ for $\vec{x} \in X$ and $\vec{x}' \in X^\perp$.

\end{enumerate}

9. $V = \mathbb{R}^3$ with $\left< \vec{v}, \vec{w} \right> = \vec{v} \bullet \vec{w}$, $X = \operatorname{span}\left\{ \left[\begin{array}{c} 1 \\ 2 \\ 0 \end{array}\right], \left[\begin{array}{c} -1 \\ 3 \\ 1 \end{array}\right] \right\}$, and $\vec{v} = \left[\begin{array}{c} 4 \\ 5 \\ 2 \end{array}\right]$.

10. $V = \operatorname{span}\left\{ 1, x, x^2 \right\}$ with $\left< p, q \right> = \sum_{n = 0}^2 p(n)q(n)$, $X$ is the subspace of polynomials $p$ with $p'(0) = 0$, and $\vec{v} = 1 + 2x + x^2$.

11. (optional) $V = \operatorname{span}\left\{ 1, \sin(x), \cos(x) \right\}$ with $\left< f, g \right> = \int_0^1 f(x)g(x)\,\text{d} x$, $X$ is the subspace of functions $f$ with $f'(0) = 0$, and $\vec{v} = \sin(x) + 2\cos(x) - 1$.

12. (optional; more recommended than the previous problem) $V = \operatorname{span}\left\{ 1, \cos(x) \right\}$ with $\left< f, g \right> = \int_0^{\pi/2} f(x)g(x)\,\text{d} x$, $X = \operatorname{span}\{1\}$, and $\vec{v} = \cos(x) - 2$.


\end{document}