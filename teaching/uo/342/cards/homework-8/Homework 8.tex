\documentclass{article}
\usepackage[utf8]{inputenc}
\usepackage[T1]{fontenc}
\usepackage{amsmath}
\usepackage{amsfonts}
\usepackage{amssymb}
\usepackage[dvipsnames]{xcolor}
\usepackage{enumitem}
\usepackage{titlesec}
\usepackage{graphicx}
\usepackage[total={6.5in, 9in}, heightrounded]{geometry}
\usepackage{hyperref}

\graphicspath{{graphics/}}
\setenumerate[0]{label=\alph*)}
\setlength{\parindent}{0pt}
\setlength{\parskip}{8pt}
\setlength\fboxsep{0pt}
\renewcommand{\baselinestretch}{1.6}
\titleformat{\section}
{\normalfont \Large \bfseries \centering}{}{0pt}{}

\newcommand{\s}[1]{{\color{violet} #1}}

\begin{document}

\Large Name: \rule{2in}{0.15mm} \hfill Homework 8 | Math 341 | Cruz Godar \vspace{4pt} \normalsize

\textit{Due Wednesday of Week 9 at the start of class}

Complete the following problems and submit them as a pdf to Canvas. 8 points are awarded for thoroughly attempting every problem, and I'll select three problems to grade on correctness for 4 points each. Enough work should be shown that there is no question about the mathematical process used to obtain your answers.

\section{Section 10}

In problems 1--5, find the determinant of the given matrix.

1. $\displaystyle \mathbf{A} = \left[\begin{array}{cc}2& -1 \\ 3& -2\end{array}\right].$

2. $\displaystyle \mathbf{B} = \left[\begin{array}{cc}1& -1 \\ 1& 3\end{array}\right].$

3. $\displaystyle \mathbf{C} = \left[\begin{array}{ccc}19& -4& 8 \\ -8& 5& -10 \\ -1& -2& 4\end{array}\right].$

4. $\displaystyle \mathbf{D} = \left[\begin{array}{ccc}2& 2& -2 \\ -3& 7& 3 \\ -5& 5& 5\end{array}\right].$

5. $\displaystyle \mathbf{E} = \left[\begin{array}{cccc}5& 6& 4& -4 \\ 3& 8& -2& 2 \\ 3& -3& 9& 2 \\ 0& 0& 0& 11\end{array}\right].$

~\\

In problems 6--10, find the eigenvalues and eigenvectors of the given matrix, and verify that the product of the eigenvalues is the determinant you found before.

6. $\mathbf{A}$ from problem 1.

7. $\mathbf{B}$ from problem 2.

8. $\mathbf{C}$ from problem 3.

9. $\mathbf{D}$ from problem 4.

10. $\mathbf{E}$ from problem 5. Hint: you may it useful to know that $(x - 11)^3 = x^3 - 33x^2 + 363x - 1331$.

~\\

11. With $\mathbf{A}$ from problem 1, find the following:

\begin{enumerate}

	\item $\displaystyle \mathbf{A}^{100} \left[\begin{array}{c}1 \\ 3\end{array}\right].$

	\item $\displaystyle \mathbf{A}^{100} \left[\begin{array}{c}-2 \\ -6\end{array}\right].$

	\item $\displaystyle \mathbf{A}^{100} \left[\begin{array}{c}0 \\ -4\end{array}\right].$

\end{enumerate}

~\\

12. A remarkable property of the determinant is that it's \textbf{multiplicative}: for any two $n \times n$ matrices $\mathbf{A}$ and $\mathbf{B}$, $\det (\mathbf{AB}) = \left( \det \mathbf{A} \right)\left( \det \mathbf{B} \right)$. Use this property to answer the following questions.

\begin{enumerate}

	\item What is $\det \mathbf{I}$? Hint: $\mathbf{IA} = \mathbf{A}$ for any matrix $\mathbf{A}$ for which the product makes sense.

	\item Let $\mathbf{A}$ be an invertible matrix. What is $\det \left(\mathbf{A}^{-1}\right)$ in terms of $\det \mathbf{A}$?

\end{enumerate}

~\\

13. Suppose $\mathbf{A}$ is an invertible matrix with eigenvectors $\mathbf{v_1}, ..., \mathbf{v_n}$ with corresponding eigenvalues $\lambda_1, ..., \lambda_n$.

\begin{enumerate}

	\item For $\mathbf{A}$ to be invertible, none of the $\lambda_i$ can be zero. Why is this?

	\item What are the eigenvectors and eigenvalues of $\mathbf{A}^{-1}$? Hint: start with $\mathbf{Av_i} = \lambda_i \mathbf{v_i}$ and find a way to introduce $\mathbf{A}^{-1}$ into the equation.

\end{enumerate}

</div>ce-tex="\vec{v} = \left[\begin{array}{c} 4 \\ 5 \\ 2 \end{array}\right]">$\vec{v} = \left[\begin{array}{c} 4 \\ 5 \\ 2 \end{array}\right]$.</span></p>10. $V = \operatorname{span}\left\{ 1, x, x^2 \right\}$ with $\left< p, q \right> = \sum_{n = 0}^2 p(n)q(n)$, $X$ is the subspace of polynomials $p$ with $p'(0) = 0$, and $\vec{v} = 1 + 2x + x^2$.

11. (optional) $V = \operatorname{span}\left\{ 1, \sin(x), \cos(x) \right\}$ with $\left< f, g \right> = \int_0^1 f(x)g(x)\,\text{d} x$, $X$ is the subspace of functions $f$ with $f'(0) = 0$, and $\vec{v} = \sin(x) + 2\cos(x) - 1$.

12. (optional; more recommended than the previous problem) $V = \operatorname{span}\left\{ 1, \cos(x) \right\}$ with $\left< f, g \right> = \int_0^{\pi/2} f(x)g(x)\,\text{d} x$, $X = \operatorname{span}\{1\}$, and $\vec{v} = \cos(x) - 2$.


\end{document}