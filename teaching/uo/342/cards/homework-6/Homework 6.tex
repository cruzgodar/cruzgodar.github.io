\documentclass{article}
\usepackage[utf8]{inputenc}
\usepackage[T1]{fontenc}
\usepackage{amsmath}
\usepackage{amsfonts}
\usepackage{amssymb}
\usepackage[dvipsnames]{xcolor}
\usepackage{enumitem}
\usepackage{titlesec}
\usepackage{graphicx}
\usepackage[total={6.5in, 9in}, heightrounded]{geometry}
\usepackage{hyperref}

\graphicspath{{graphics/}}
\setenumerate[0]{label=\alph*)}
\setlength{\parindent}{0pt}
\setlength{\parskip}{8pt}
\setlength\fboxsep{0pt}
\renewcommand{\baselinestretch}{1.6}
\titleformat{\section}
{\normalfont \Large \bfseries \centering}{}{0pt}{}

\newcommand{\s}[1]{{\color{violet} #1}}

\begin{document}

\Large Name: \rule{2in}{0.15mm} \hfill Homework 6 | Math 341 | Cruz Godar \vspace{4pt} \normalsize

\textit{Due Wednesday of Week 7 at the start of class}

Complete the following problems and submit them as a pdf to Canvas. 8 points are awarded for thoroughly attempting every problem, and I'll select three problems to grade on correctness for 4 points each. Enough work should be shown that there is no question about the mathematical process used to obtain your answers.

\section{Section 5}

In problems 1--4, find the least-squares solution $\vec{x}'$ to the matrix equation $A\vec{x} = \vec{b}$ and the distance between $A\vec{x}'$ and $\vec{b}$. Also determine if that least-squares solution is unique.

1. $\left[\begin{array}{cc} 1& 1 \\ 2& 3 \\ 0& 2 \end{array}\right]\vec{x} = \left[\begin{array}{c} 1 \\ 0 \\ 2 \end{array}\right]$.

2. $\left[\begin{array}{ccc} 1& 2& 3 \\ -1& 3& 2 \\ 0& 2& 2 \end{array}\right]\vec{x} = \left[\begin{array}{c} 3 \\ 1 \\ 6 \end{array}\right]$.

3. $\left[\begin{array}{ccc} 1& 2& 1 \\ 4& 2& -1 \\ 3& 0& 1 \end{array}\right]\vec{x} = \left[\begin{array}{c} 8 \\ 5 \\ 6 \end{array}\right]$.

4. $\left[\begin{array}{ccc} 1& 0& 2 \\ 2& 0& 1 \\ 1& 1& 2 \\ 3& 1& 1 \end{array}\right]\vec{x} = \left[\begin{array}{c} 1 \\ 1 \\ -2 \\ -3 \end{array}\right]$.

~\\

5. Let $A$ be an $m \times n$ unitary matrix (i.e. not necessarily square). What is the least-squares solution to $A\vec{x} = \vec{b}$ in terms of $A$ and $\vec{b}$?

\section{Section 6}

In problems 6--8, compute $\left< \vec{v}, \vec{w} \right>$, $\left| \left| \vec{v} \right| \right|$, $\left| \left| \vec{w} \right| \right|$, and the angle between $\vec{v}$ and $\vec{w}$ for the given inner product $V$.

6. $\vec{v} = 1 + 2x$ and $\vec{w} = 2x + x^2 - x^3$ for $V = \operatorname{span}\left\{ 1, x, x^2, x^3 \right\}$ with the inner product

$$
	\left< p, q \right> = \sum_{n = 0}^3 p(n)q(n).
$$

7. $\vec{v} = \left[\begin{array}{c} 1 \\ 2 \\ -1 \end{array}\right]$ and $\vec{w} = \left[\begin{array}{c} 2 \\ 1 \\ 0 \end{array}\right]$ for $V = \mathbb{R}^3$ with the inner product

$$
	\left< \vec{v}, \vec{w} \right> = 2v_1w_1 + v_2w_2 + 50v_3w_3.
$$

8. $\vec{v}(x) = e^x$ and $\vec{w}(x) = x$ for $V = C[0, 1]$ with the inner product

$$
	\left< f, g \right> = \int_0^1 f(x)g(x)\,\text{d} x.
$$

~\\

9. Let $V = C[0, 1]$. For each of the following possible inner product formulas, give an example showing that it does not define an inner product.

\begin{enumerate}

	\item $\displaystyle \left< f, g \right> = \int_0^1 |f(x) + g(x)|\,\text{d} x.$

	\item $\displaystyle \left< f, g \right> = \int_0^1 (f(x)g(x))^2\,\text{d} x.$

	\item $\displaystyle \left< f, g \right> = \int_0^1 f(g(x))\,\text{d} x.$

	\item $\displaystyle \left< f, g \right> = \int_{1/4}^{3/4} f(x)g(x)\,\text{d} x.$

\end{enumerate}


\end{document}