\documentclass{article}
\usepackage[utf8]{inputenc}
\usepackage[T1]{fontenc}
\usepackage{amsmath}
\usepackage{amsfonts}
\usepackage{amssymb}
\usepackage[dvipsnames]{xcolor}
\usepackage{enumitem}
\usepackage{titlesec}
\usepackage{graphicx}
\usepackage[total={6.5in, 9in}, heightrounded]{geometry}
\usepackage{hyperref}

\hypersetup
{
	colorlinks = true,
	allcolors = OliveGreen
}

\graphicspath{{graphics/}}
\setenumerate[0]{label=\alph*)}
\setlength{\parindent}{0pt}
\setlength{\parskip}{8pt}
\setlength\fboxsep{0pt}
\renewcommand{\baselinestretch}{1.6}
\titleformat{\section}
{\normalfont \Large \bfseries \centering}{}{0pt}{}

\newcommand{\s}[1]{{\color{violet} #1}}

\begin{document}

\Large Name: \rule{2in}{0.15mm} \hfill Homework 10 | Math 1180 | Cruz Godar \vspace{4pt} \normalsize

\textit{Due Sunday, November 7th at 11:59 PM}

Complete the following problems and submit them as a pdf to Gradescope. You should show enough work that there is no question about the mathematical process used to obtain your answers, and so that your peers in the class could easily follow along. I encourage you to collaborate with your classmates, so long as you write up your solutions independently. If you collaborate with any classmates, please include a statement on your assignment acknowledging with whom you collaborated.

~\\

In problems 1--6, determine if the linear transformation $T$ is one-to-one, if it is onto, and if it is invertible. If it is invertible, find the inverse transformation.

1. $T: \mathbb{R}^2 \to \mathbb{R}^2$ defined by $T\left(\left[\begin{array}{c} x \\ y \end{array}\right]\right) = \left[\begin{array}{c} 2x + y \\ x + 2y\end{array}\right]$.

2. $T: \mathbb{R}^3 \to \mathbb{R}^2$ defined by $T\left(\left[\begin{array}{c} x \\ y \\ z \end{array}\right]\right) = \left[\begin{array}{c} x \\ y + z \end{array}\right]$.

3. $T: \mathbb{R}^2 \to \mathbb{R}^2$, where

$$
	T\left(\left[\begin{array}{c} 2 \\ 0 \end{array}\right]\right) = \left[\begin{array}{c} 2 \\ 4 \end{array}\right] \qquad T\left(\left[\begin{array}{c} 4 \\ 6 \end{array}\right]\right) = \left[\begin{array}{c} 4 \\ -10 \end{array}\right].
$$

4. $T: \mathbb{R}^3 \to \mathbb{R}^3$, where

$$
	T\left(\left[\begin{array}{c} 1 \\ 1 \\ 1 \end{array}\right]\right) = \left[\begin{array}{c} 3 \\ 7 \\ 4 \end{array}\right] \qquad T\left(\left[\begin{array}{c} 1 \\ 2 \\ 0 \end{array}\right]\right) = \left[\begin{array}{c} 1 \\ -1 \\ 3 \end{array}\right] \qquad T\left(\left[\begin{array}{c} 0 \\ 1 \\ 3 \end{array}\right]\right) = \left[\begin{array}{c} 6 \\ 16 \\ 7 \end{array}\right].
$$

5. $T: \mathbb{R}^3 \to \mathbb{R}$, where

$$
	T\left(\left[\begin{array}{c} 1 \\ -1 \\ 1 \end{array}\right]\right) = -3 \qquad T\left(\left[\begin{array}{c} 1 \\ 0 \\ 1 \end{array}\right]\right) = 1 \qquad T\left(\left[\begin{array}{c} 0 \\ 1 \\ 1 \end{array}\right]\right) = 4.
$$

6. $T: \mathbb{R}^5 \to \mathbb{R}^2$, where

$$
	T\left(\left[\begin{array}{c} 1 \\ 0 \\ 2 \\ 3 \\ 0 \end{array}\right]\right) = \left[\begin{array}{c} 1 \\ 1 \end{array}\right] \qquad T\left(\left[\begin{array}{c} 2 \\ 1 \\ 2 \\ -1 \\ 3 \end{array}\right]\right) = \left[\begin{array}{c} 2 \\ 0 \end{array}\right].
$$

~\\

7. Let $R : \mathbb{R}^2 \to \mathbb{R}^2$ be a function (not necessarily a linear transformation) defined by rotating its inputs $90^\circ$ counterclockwise. For example, $R\left( \left[\begin{array}{c} 1 \\ 0 \end{array}\right] \right) = \left[\begin{array}{c} 0 \\ 1 \end{array}\right]$ and $R\left( \left[\begin{array}{c} \frac{1}{2} \\ \frac{\sqrt{3}}{2} \end{array}\right] \right) = \left[\begin{array}{c} -\frac{\sqrt{3}}{2} \\ \frac{1}{2} \end{array}\right]$.

\begin{enumerate}

	\item Explain with a picture why $R$ is in fact a linear transformation, i.e. $R(\vec{v_1} + \vec{v_2}) = R(\vec{v_1}) + R(\vec{v_2})$ and $R(c\vec{v}) = cR(\vec{v})$.

	\item Find the matrix for $R$.

	\item Let $R_\theta : \mathbb{R}^2 \to \mathbb{R}^2$ be the linear transformation that rotates its inputs an angle $\theta$ counterclockwise, where $\theta$ is a variable. Find a matrix for $R_\theta$.

\end{enumerate}

~\\

In problems 8--12, compute the determinant of the given matrix.

8. $\displaystyle A = \left[\begin{array}{cc} 2& 3 \\ -3& 1 \end{array}\right]$.

9. $\displaystyle B = \left[\begin{array}{ccc} 2& 3& 0 \\ -3& 1& -1 \\ 1& 1& 1 \end{array}\right]$.

10. $\displaystyle C = \left[\begin{array}{ccc} 1& 1& -3 \\ 0& 1& 3 \\ 2& -1& -15 \end{array}\right]$.

11. $\displaystyle D = \left[\begin{array}{ccc} 1& 2& 3 \\ 4& 5& 6 \end{array}\right]$.

12. $\displaystyle E = \left[\begin{array}{cccc} 1& 2& 3& 4 \\ 3& -1& 0& 3 \\ 2& 0& 1& 2\\ -1& -3& -7& 2 \end{array}\right]$.

~\\

13. For each of the matrices $A$--$E$ in problems 8--12, classify it as invertible or noninvertible based on its determinant.

14. Let $A$ be the matrix from problem 8. Sketch a picture of the unit square in $\mathbb{R}^2$ and its image under the linear operator corresponding to $A$. Verify that the area of that image is $|\det A|$ times the area of the unit square (i.e. $1$).

15. We can use the multiplicativity of the determinant to show some nice facts about the determinants of inverse matrices.

\begin{enumerate}

	\item What is $\det I$?

	\item Let $A$ be an invertible matrix. Using part a), find $\det A^{-1}$ in terms of $\det A$.

\end{enumerate}


\end{document}