\documentclass{article}
\usepackage[utf8]{inputenc}
\usepackage[T1]{fontenc}
\usepackage{amsmath}
\usepackage{amsfonts}
\usepackage{amssymb}
\usepackage[dvipsnames]{xcolor}
\usepackage{enumitem}
\usepackage{titlesec}
\usepackage{graphicx}
\usepackage[total={6.5in, 9in}, heightrounded]{geometry}
\usepackage{hyperref}

\hypersetup
{
	colorlinks = true,
	allcolors = OliveGreen
}

\graphicspath{{graphics/}}
\setenumerate[0]{label=\alph*)}
\setlength{\parindent}{0pt}
\setlength{\parskip}{8pt}
\setlength\fboxsep{0pt}
\renewcommand{\baselinestretch}{1.6}
\titleformat{\section}
{\normalfont \Large \bfseries \centering}{}{0pt}{}

\newcommand{\s}[1]{{\color{violet} #1}}

\begin{document}

\Large Name: \rule{2in}{0.15mm} \hfill Homework 9 | Math 1180 | Cruz Godar \vspace{4pt} \normalsize

\textit{Due Monday, November 17th at 11:59 PM}

Complete the following problems and submit them as a pdf to Gradescope. You should show enough work that there is no question about the mathematical process used to obtain your answers, and so that your peers in the class could easily follow along. I encourage you to collaborate with your classmates, so long as you write up your solutions independently. If you collaborate with any classmates, please include a statement on your assignment acknowledging with whom you collaborated.

On this homework, please do row reduction by hand. However, please do check your answers with technology for all of the problems! I recommend \href{https://www.wolframalpha.com/}{Wolfram Alpha} for this purpose, where you can ask questions of the form "Row reduce {{1, 2, 3}, {4, 5, 6}}" to row reduce the matrix

\begin{align*}
	\left[\begin{array}{ccc} 1& 2& 3 \\ 4& 5& 6 \end{array}\right].
\end{align*}

Whatever you choose to use, please use something deterministic; at the time of writing, LLMs are notoriously bad at not dropping numbers and signs with this much arithmetic.

~\\

At this point, please feel free to do row reduction with technology --- note that you should still be doing the setup and all the rest of the problem on your own.

In problems 1--3, determine if the vectors are linearly dependent or independent. If they are dependent, find a linear combination equal to $\vec{0}$.

1. $\vec{v_1} = \left[\begin{array}{c} 1 \\ 2 \\ 3 \end{array}\right]$, $\vec{v_2} = \left[\begin{array}{c} 1 \\ 0 \\ 2 \end{array}\right]$, $\vec{v_3} = \left[\begin{array}{c} 1 \\ 4 \\ 4 \end{array}\right]$.

2. $\vec{v_1} = \left[\begin{array}{c} 1 \\ 1 \\ 2 \end{array}\right]$, $\vec{v_2} = \left[\begin{array}{c} 2 \\ 0 \\ 3 \end{array}\right]$, $\vec{v_3} = \left[\begin{array}{c} 4 \\ - 1 \\ 2 \end{array}\right]$

3. $\vec{v_1} = \left[\begin{array}{c} 1 \\ 5 \\ 6 \\ -10 \\ \frac{17}{2} \end{array}\right]$, $\vec{v_2} = \left[\begin{array}{c} 3 \\ 1 \\ 7 \\ -100 \\ 0\end{array}\right]$, $\vec{v_3} = \left[\begin{array}{c} 0 \\ 0 \\ 0 \\ 0 \\ 0 \end{array}\right]$, $\vec{v_4} = \left[\begin{array}{c} 4 \\ 1 \\ 2 \\ 0 \\ 89 \end{array}\right]$. (Please do this without technology!)

~\\

In problems 4--5, express $\vec{v}$ as a linear combination of the $\vec{u}_i$ or show it's impossible.

4. $\vec{v} = \left[\begin{array}{c} 1 \\ 2 \\ 3 \end{array}\right]$, $\quad \vec{u_1} = \left[\begin{array}{c} 1 \\ 2 \\ 2 \end{array}\right]$, $\vec{u_2} = \left[\begin{array}{c} 2 \\ -1 \\ 0 \end{array}\right]$, $\vec{u_3} = \left[\begin{array}{c} 3 \\ 1 \\ 1 \end{array}\right]$.

5. $\vec{v} = \left[\begin{array}{c} 1 \\ 2 \\ 3 \end{array}\right]$, $\quad \vec{u_1} = \left[\begin{array}{c} -1 \\ 3 \\ 0 \end{array}\right]$, $\vec{u_2} = \left[\begin{array}{c} 1 \\ 1 \\ 1 \end{array}\right]$, $\vec{u_3} = \left[\begin{array}{c} 2 \\ 6 \\ 3 \end{array}\right]$.

~\\

6. Let $\displaystyle \vec{v} = \left[\begin{array}{c} 1 \\ 3 \end{array}\right]$ and $\displaystyle \vec{w} = \left[\begin{array}{c} -1 \\ 1 \end{array}\right]$.

\begin{enumerate}

	\item Draw $\vec{v}$ and $\vec{w}$ in the plane.

	\item Draw a vector that is linearly dependent with $\vec{v}$, but linearly independent with $\vec{w}$.

	\item Draw a vector that is linearly dependent with both $\vec{v}$ and $\vec{w}$, but not with either of them alone.

\end{enumerate}


\end{document}